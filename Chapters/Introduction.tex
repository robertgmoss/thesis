%************************************************
\chapter{Introduction}\label{ch:introduction}
%************************************************
%\begin{flushleft}{\slshape    
%    At the time, Nixon was normalizing relations with China.  
%    I figured that if he could normalize relations, then so could I} \\ \medskip
%    --- Edgar Codd
%\end{flushleft}
%
Many modern data collections defy the traditional storage mechanisms of primary key based relational databases. With no natural ordering, it becomes either impossible or undesirable to index collections of this type.  For example, with a corpus of semi-structured data, or images on the world-wide-web, exact-match solutions, which have typified traditional databases, provide inadequate and often meaningless solutions.

As the number of such collections grow, there becomes a pressing need to address the problem of search.  In many cases, similarity searching provides the answer. 

%In this introductory chapter, we present general metric spaces as an abstraction for working in similarity search; we examine techniques for partitioning space to assist with indexing; and thereafter, take a look at common search query types.

% ********************** %
\section{Thesis outline}
% ********************** %
Having introduced the notion of similarity search in general terms, we now outline what follows in this thesis.  Part 1 introduces the mathematical preliminaries that underpin the work in subsequent parts.  It consists of a single chapter which covers aspects of distance, probability, and information theory that are pertinent.  This is followed in Part 2 with three chapters that deal with combining distance, probability and information theory. First, by introducing various methods of computing distance with information in chapter \ref{ch:information_distances}; Secondly, by narrowing the focus specifically to Structural Entropic Distance in chapter \ref{ch:multiway_structural_entropic_distance} and generalising it to the notion of multi-way distance; and thirdly, by exploring efficient means of evaluating Structural Entropic Distance.  After this, follows the final part, Part 3, which investigates the use of Structural Entropic Distance.  It consists three further chapters: Chapter \ref{ch:similarity_search}, investigates the performance of Structural Entropic Distance as a distance metric used with Similarity Search indexing techniques; chapter \ref{ch:information_retrieval}, explores a modification to Structural Entropic Distance to perform the traditional Information Retrieval task of searching for documents relevant to a set of keywords.  Finally, in chapter \ref{ch:clustering_classification_and_outlier_detection} we make use of the multi-way distance generalisation to perform classic data-mining techniques such as clustering, classification, and outlier detection.